\section{总结及展望}
% 一些关于序号的设置
\setcounter{figure}{0}
\textbf{目标检测算法出现了以下几种:}

1、传统的目标检测算法:Cascade\cite{repid} + Haar\cite{haar} / SVM\cite{svm} + HOG / DPM ;

2、候选窗+深度学习分类:通过提取候选区域,并对相应区域进行以深度学习方法为主的分类的方案,如:RCNN / SPP-net/ Fast-RCNN / Faster-RCNN  / R-FCN\cite{rfcn}系列方法;

3、基于深度学习的回归方法:YOLO / SSD / DenseBox\cite{densebox}等方法;

4、结合RNN算法的RRC Detection;结合DPM的Deformable CNN\cite{deformable}等方法;

\textbf{基于深度学习方法的几个可能的方向:}

1、从原始图像、低层的Feature Map层,以及高层的语义层获取更多的信息,从而得到对目标Bounding Box的更准确的估计;

2、对Bounding Box的估计可以结合图片的一些由粗到细(coarse-to-fine)的分割信息;

3、对bounding box的估计需要引入更多的局部的content的信息;

4、目标检测数据集的标注难度非常大,如何把其他如classfication领域学习到的知识用于检测当中,甚至是将classification的数据和检测数据集做co-training(如YOLO9000)的方式,可以从数据层面获得更多的信息;

5、RRC,deformable cnn中卷积和其他的很好的图片的操作、机器学习的思想的结合未来也有很大的空间;

6、语意信息和分割的结合,可能能够为目标检测提供更多的有用的信息;

7、场景信息也会为目标检测提供更多信息;比如天空不会出现汽车等等。