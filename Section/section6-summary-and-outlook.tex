\section{总结及展望}
% 一些关于序号的设置
\setcounter{figure}{0}
目前业界出现的目标检测算法有一下几种:

1、传统的目标检测算法:Cascade + Haar / SVM + HOG / DPM ;

2、候选窗+深度学习分类:通过提取候选区域,并对相应区域进行以深度学习方法为主的分类的方案,如:RCNN / SPP-net/ Fast-RCNN / Faster-RCNN  / R-FCN 系列方法;

3、基于深度学习的回归方法:YOLO / SSD / DenseBox 等方法;

4、结合RNN算法的RRC detection;结合DPM的Deformable CNN等方法;

基于深度学习方法的几个可能的方向:

1、从原始图像、低层的feature map层,以及高层的语义层获取更多的信息,从而得到对目标bounding box的更准确的估计;

2、对bounding box的估计可以结合图片的一些由粗到细(coarse-to-fine)的分割信息;

3、对bounding box的估计需要引入更多的局部的content的信息;

4、目标检测数据集的标注难度非常大,如何把其他如classfication领域学习到的知识用于检测当中,甚至是将classification的数据和检测数据集做co-training(如YOLO9000)的方式,可以从数据层面获得更多的信息;

5、更好的启发式的学习方式,人在识别物体的时候,第一次可能只是知道这是一个单独的物体,也知道bounding box,但是不知道类别;当人类通过其他渠道学习到类别的时候,下一次就能够识别了;目标检测也是如此,我们不可能标注所有的物体的类别,但是如何将这种快速学习的机制引入,也是一个问题;

6、RRC,deformable cnn中卷积和其他的很好的图片的操作、机器学习的思想的结合未来也有很大的空间;

7、语意信息和分割的结合,可能能够为目标检测提供更多的有用的信息;

8、场景信息也会为目标检测提供更多信息;比如天空不会出现汽车等等。