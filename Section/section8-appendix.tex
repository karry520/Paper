\section*{附录}
\addcontentsline{toc}{section}{附录}
\subsection*{源代码}
\subsubsection*{datagen.py}
\begin{lstlisting}[caption={datagen.py}]
'''Load image/class/box from a annotation file.

The annotation file is organized as:
image_name #obj xmin ymin xmax ymax class_index ..
'''
from __future__ import print_function

import os
import sys
import os.path

import random
import numpy as np

import torch
import torch.utils.data as data
import torchvision.transforms as transforms

from encoder import DataEncoder
from PIL import Image, ImageOps


class ListDataset(data.Dataset):
img_size = 300

def __init__(self, root, list_file, train, transform):
'''
Args:
root: (str) ditectory to images.
list_file: (str) path to index file.
train: (boolean) train or test.
transform: ([transforms]) image transforms.
'''
self.root = root
self.train = train
self.transform = transform

self.fnames = []
self.boxes = []
self.labels = []

self.data_encoder = DataEncoder()

with open(list_file) as f:
lines = f.readlines()
self.num_samples = len(lines)

for line in lines:
splited = line.strip().split()
self.fnames.append(splited[0])

num_objs = int(splited[1])
box = []
label = []
for i in range(num_objs):
xmin = splited[2+5*i]
ymin = splited[3+5*i]
xmax = splited[4+5*i]
ymax = splited[5+5*i]
c = splited[6+5*i]
box.append([float(xmin),float(ymin),float(xmax),float(ymax)])
label.append(int(c))
self.boxes.append(torch.Tensor(box))
self.labels.append(torch.LongTensor(label))

def __getitem__(self, idx):
'''Load a image, and encode its bbox locations and class labels.

Args:
idx: (int) image index.

Returns:
img: (tensor) image tensor.
loc_target: (tensor) location targets, sized [8732,4].
conf_target: (tensor) label targets, sized [8732,].
'''
# Load image and bbox locations.
fname = self.fnames[idx]
img = Image.open(os.path.join(self.root, fname))
boxes = self.boxes[idx].clone()
labels = self.labels[idx]

# Data augmentation while training.
if self.train:
img, boxes = self.random_flip(img, boxes)
img, boxes, labels = self.random_crop(img, boxes, labels)

# Scale bbox locaitons to [0,1].
w,h = img.size
boxes /= torch.Tensor([w,h,w,h]).expand_as(boxes)

img = img.resize((self.img_size,self.img_size))
img = self.transform(img)

# Encode loc & conf targets.
loc_target, conf_target = self.data_encoder.encode(boxes, labels)
return img, loc_target, conf_target

def random_flip(self, img, boxes):
'''Randomly flip the image and adjust the bbox locations.

For bbox (xmin, ymin, xmax, ymax), the flipped bbox is:
(w-xmax, ymin, w-xmin, ymax).

Args:
img: (PIL.Image) image.
boxes: (tensor) bbox locations, sized [#obj, 4].

Returns:
img: (PIL.Image) randomly flipped image.
boxes: (tensor) randomly flipped bbox locations, sized [#obj, 4].
'''
if random.random() < 0.5:
img = img.transpose(Image.FLIP_LEFT_RIGHT)
w = img.width
xmin = w - boxes[:,2]
xmax = w - boxes[:,0]
boxes[:,0] = xmin
boxes[:,2] = xmax
return img, boxes

def random_crop(self, img, boxes, labels):
'''Randomly crop the image and adjust the bbox locations.

For more details, see 'Chapter2.2: Data augmentation' of the paper.

Args:
img: (PIL.Image) image.
boxes: (tensor) bbox locations, sized [#obj, 4].
labels: (tensor) bbox labels, sized [#obj,].

Returns:
img: (PIL.Image) cropped image.
selected_boxes: (tensor) selected bbox locations.
labels: (tensor) selected bbox labels.
'''
imw, imh = img.size
while True:
min_iou = random.choice([None, 0.1, 0.3, 0.5, 0.7, 0.9])
if min_iou is None:
return img, boxes, labels

for _ in range(100):
w = random.randrange(int(0.1*imw), imw)
h = random.randrange(int(0.1*imh), imh)

if h > 2*w or w > 2*h:
continue

x = random.randrange(imw - w)
y = random.randrange(imh - h)
roi = torch.Tensor([[x, y, x+w, y+h]])

center = (boxes[:,:2] + boxes[:,2:]) / 2  # [N,2]
roi2 = roi.expand(len(center), 4)  # [N,4]
mask = (center > roi2[:,:2]) & (center < roi2[:,2:])  # [N,2]
mask = mask[:,0] & mask[:,1]  #[N,]
if not mask.any():
continue

selected_boxes = boxes.index_select(0, mask.nonzero().squeeze(1))

iou = self.data_encoder.iou(selected_boxes, roi)
if iou.min() < min_iou:
continue

img = img.crop((x, y, x+w, y+h))
selected_boxes[:,0].add_(-x).clamp_(min=0, max=w)
selected_boxes[:,1].add_(-y).clamp_(min=0, max=h)
selected_boxes[:,2].add_(-x).clamp_(min=0, max=w)
selected_boxes[:,3].add_(-y).clamp_(min=0, max=h)
return img, selected_boxes, labels[mask]

def __len__(self):
return self.num_samples
\end{lstlisting}
\subsubsection*{ssd.py}
\begin{lstlisting}
import math
import itertools

import torch
import torch.nn as nn
import torch.nn.functional as F
import torch.nn.init as init

from torch.autograd import Variable

from multibox_layer import MultiBoxLayer


class L2Norm2d(nn.Module):
'''L2Norm layer across all channels.'''
def __init__(self, scale):
super(L2Norm2d, self).__init__()
self.scale = scale

def forward(self, x, dim=1):
'''out = scale * x / sqrt(\sum x_i^2)'''
return self.scale * x * x.pow(2).sum(dim).clamp(min=1e-12).
	rsqrt().expand_as(x)


class SSD300(nn.Module):
input_size = 300

def __init__(self):
super(SSD300, self).__init__()

# model
self.base = self.VGG16()
self.norm4 = L2Norm2d(20)

self.conv5_1 = nn.Conv2d(512, 512, kernel_size=3, padding=1, dilation=1)
self.conv5_2 = nn.Conv2d(512, 512, kernel_size=3, padding=1, dilation=1)
self.conv5_3 = nn.Conv2d(512, 512, kernel_size=3, padding=1, dilation=1)

self.conv6 = nn.Conv2d(512, 1024, kernel_size=3, padding=6, dilation=6)

self.conv7 = nn.Conv2d(1024, 1024, kernel_size=1)

self.conv8_1 = nn.Conv2d(1024, 256, kernel_size=1)
self.conv8_2 = nn.Conv2d(256, 512, kernel_size=3, padding=1, stride=2)

self.conv9_1 = nn.Conv2d(512, 128, kernel_size=1)
self.conv9_2 = nn.Conv2d(128, 256, kernel_size=3, padding=1, stride=2)

self.conv10_1 = nn.Conv2d(256, 128, kernel_size=1)
self.conv10_2 = nn.Conv2d(128, 256, kernel_size=3)

self.conv11_1 = nn.Conv2d(256, 128, kernel_size=1)
self.conv11_2 = nn.Conv2d(128, 256, kernel_size=3)

# multibox layer
self.multibox = MultiBoxLayer()

def forward(self, x):
hs = []
h = self.base(x)
hs.append(self.norm4(h))  # conv4_3

h = F.max_pool2d(h, kernel_size=2, stride=2, ceil_mode=True)

h = F.relu(self.conv5_1(h))
h = F.relu(self.conv5_2(h))
h = F.relu(self.conv5_3(h))
h = F.max_pool2d(h, kernel_size=3, padding=1, stride=1, ceil_mode=True)

h = F.relu(self.conv6(h))
h = F.relu(self.conv7(h))
hs.append(h)  # conv7

h = F.relu(self.conv8_1(h))
h = F.relu(self.conv8_2(h))
hs.append(h)  # conv8_2

h = F.relu(self.conv9_1(h))
h = F.relu(self.conv9_2(h))
hs.append(h)  # conv9_2

h = F.relu(self.conv10_1(h))
h = F.relu(self.conv10_2(h))
hs.append(h)  # conv10_2

h = F.relu(self.conv11_1(h))
h = F.relu(self.conv11_2(h))
hs.append(h)  # conv11_2

loc_preds, conf_preds = self.multibox(hs)
return loc_preds, conf_preds

def VGG16(self):
'''VGG16 layers.'''
cfg = [64, 64, 'M', 128, 128, 'M', 256, 256, 256, 'M', 512, 512, 512]
layers = []
in_channels = 3
for x in cfg:
if x == 'M':
layers += [nn.MaxPool2d(kernel_size=2, stride=2, ceil_mode=True)]
else:
layers += [nn.Conv2d(in_channels, x, kernel_size=3, padding=1),
nn.ReLU(True)]
in_channels = x
return nn.Sequential(*layers)
\end{lstlisting}

\subsubsection*{multibox\_layer.py}
\begin{lstlisting}
from __future__ import print_function

import math

import torch
import torch.nn as nn
import torch.nn.init as init
import torch.nn.functional as F

from torch.autograd import Variable


class MultiBoxLayer(nn.Module):
num_classes = 21
num_anchors = [4,6,6,6,4,4]
in_planes = [512,1024,512,256,256,256]

def __init__(self):
super(MultiBoxLayer, self).__init__()

self.loc_layers = nn.ModuleList()
self.conf_layers = nn.ModuleList()
for i in range(len(self.in_planes)):
self.loc_layers.append(nn.Conv2d(self.in_planes[i],
 self.num_anchors[i]*4, kernel_size=3, padding=1))
self.conf_layers.append(nn.Conv2d(self.in_planes[i],
 self.num_anchors[i]*21, kernel_size=3, padding=1))

def forward(self, xs):
'''
Args:
xs: (list) of tensor containing intermediate layer outputs.

Returns:
loc_preds: (tensor) predicted locations, sized [N,8732,4].
conf_preds: (tensor) predicted class confidences,
 sized [N,8732,21].
'''
y_locs = []
y_confs = []
for i,x in enumerate(xs):
y_loc = self.loc_layers[i](x)
N = y_loc.size(0)
y_loc = y_loc.permute(0,2,3,1).contiguous()
y_loc = y_loc.view(N,-1,4)
y_locs.append(y_loc)

y_conf = self.conf_layers[i](x)
y_conf = y_conf.permute(0,2,3,1).contiguous()
y_conf = y_conf.view(N,-1,21)
y_confs.append(y_conf)

loc_preds = torch.cat(y_locs, 1)
conf_preds = torch.cat(y_confs, 1)
return loc_preds, conf_preds
\end{lstlisting}

\subsubsection*{multibox\_loss.py}
\begin{lstlisting}
from __future__ import print_function

import math

import torch
import torch.nn as nn
import torch.nn.init as init
import torch.nn.functional as F

from torch.autograd import Variable


class MultiBoxLoss(nn.Module):
num_classes = 21

def __init__(self):
super(MultiBoxLoss, self).__init__()

def cross_entropy_loss(self, x, y):
'''Cross entropy loss w/o averaging across all samples.

Args:
x: (tensor) sized [N,D].
y: (tensor) sized [N,].

Return:
(tensor) cross entroy loss, sized [N,].
'''
xmax = x.data.max()
log_sum_exp = torch.log(torch.sum(torch.exp(x-xmax), 1)) + xmax
return log_sum_exp - x.gather(1, y.view(-1,1))

def test_cross_entropy_loss(self):
a = Variable(torch.randn(10,4))
b = Variable(torch.ones(10).long())
loss = self.cross_entropy_loss(a,b)
print(loss.mean())
print(F.cross_entropy(a,b))

def hard_negative_mining(self, conf_loss, pos):
'''Return negative indices that is 3x the 
number as postive indices.

Args:
conf_loss: (tensor) cross entroy loss 
between conf_preds and conf_targets, sized [N*8732,].
pos: (tensor) positive(matched) box indices, sized [N,8732].

Return:
(tensor) negative indices, sized [N,8732].
'''
batch_size, num_boxes = pos.size()

conf_loss[pos] = 0  # set pos boxes = 0, 
the rest are neg conf_loss
conf_loss = conf_loss.view(batch_size, -1)  # [N,8732]

_,idx = conf_loss.sort(1, descending=True)  # sort by neg conf_loss
_,rank = idx.sort(1)  # [N,8732]

num_pos = pos.long().sum(1)  # [N,1]
num_neg = torch.clamp(3*num_pos, max=num_boxes-1)  # [N,1]

neg = rank < num_neg.expand_as(rank)  # [N,8732]
return neg

def forward(self, loc_preds, loc_targets, conf_preds, conf_targets):
'''Compute loss between (loc_preds, loc_targets) and (conf_preds, conf_targets).

Args:
loc_preds: (tensor) predicted locations, sized [batch_size, 8732, 4].
loc_targets: (tensor) encoded target locations, sized [batch_size, 8732, 4].
conf_preds: (tensor) predicted class confidences, sized [batch_size, 8732, num_classes].
conf_targets: (tensor) encoded target classes, sized [batch_size, 8732].

loss:
(tensor) loss = SmoothL1Loss(loc_preds, loc_targets) + CrossEntropyLoss(conf_preds, conf_targets).
'''
batch_size, num_boxes, _ = loc_preds.size()

pos = conf_targets>0  # [N,8732], pos means the box matched.
num_matched_boxes = pos.data.long().sum()
if num_matched_boxes == 0:
return Variable(torch.Tensor([0]))

################################################################
# loc_loss = SmoothL1Loss(pos_loc_preds, pos_loc_targets)
################################################################
pos_mask = pos.unsqueeze(2).expand_as(loc_preds)    # [N,8732,4]
pos_loc_preds = loc_preds[pos_mask].view(-1,4)      # [#pos,4]
pos_loc_targets = loc_targets[pos_mask].view(-1,4)  # [#pos,4]
loc_loss = F.smooth_l1_loss(pos_loc_preds, pos_loc_targets, size_average=False)

################################################################
# conf_loss = CrossEntropyLoss(pos_conf_preds, pos_conf_targets)
#           + CrossEntropyLoss(neg_conf_preds, neg_conf_targets)
################################################################
conf_loss = self.cross_entropy_loss(conf_preds.view(
-1,self.num_classes), 
conf_targets.view(-1))  # [N*8732,]
neg = self.hard_negative_mining(conf_loss, pos)    # [N,8732]

pos_mask = pos.unsqueeze(2).expand_as(conf_preds)  # [N,8732,21]
neg_mask = neg.unsqueeze(2).expand_as(conf_preds)  # [N,8732,21]
mask = (pos_mask+neg_mask).gt(0)

pos_and_neg = (pos+neg).gt(0)
preds = conf_preds[mask].view(-1,self.num_classes)  # [#pos+#neg,21]
targets = conf_targets[pos_and_neg]                 # [#pos+#neg,]
conf_loss = F.cross_entropy(preds, targets, size_average=False)

loc_loss /= num_matched_boxes
conf_loss /= num_matched_boxes
print('%f %f' % (loc_loss.data[0], conf_loss.data[0]), end=' ')
return loc_loss + conf_loss
\end{lstlisting}
\subsubsection*{encoder.py}
\begin{lstlisting}
'''Encode target locations and labels.'''
import torch

import math
import itertools

class DataEncoder:
def __init__(self):
'''Compute default box sizes with scale and aspect transform.'''
scale = 300.
steps = [s / scale for s in (8, 16, 32, 64, 100, 300)]
sizes = [s / scale for s in (30, 60, 111, 162, 213, 264, 315)]
aspect_ratios = ((2,), (2,3), (2,3), (2,3), (2,), (2,))
feature_map_sizes = (38, 19, 10, 5, 3, 1)

num_layers = len(feature_map_sizes)

boxes = []
for i in range(num_layers):
fmsize = feature_map_sizes[i]
for h,w in itertools.product(range(fmsize), repeat=2):
cx = (w + 0.5)*steps[i]
cy = (h + 0.5)*steps[i]

s = sizes[i]
boxes.append((cx, cy, s, s))

s = math.sqrt(sizes[i] * sizes[i+1])
boxes.append((cx, cy, s, s))

s = sizes[i]
for ar in aspect_ratios[i]:
boxes.append((cx, cy, s * math.sqrt(ar), s / math.sqrt(ar)))
boxes.append((cx, cy, s / math.sqrt(ar), s * math.sqrt(ar)))

self.default_boxes = torch.Tensor(boxes)

def iou(self, box1, box2):
'''Compute the intersection over union of 
two set of boxes, each box is [x1,y1,x2,y2].

Args:
box1: (tensor) bounding boxes, sized [N,4].
box2: (tensor) bounding boxes, sized [M,4].

Return:
(tensor) iou, sized [N,M].
'''
N = box1.size(0)
M = box2.size(0)

lt = torch.max(
box1[:,:2].unsqueeze(1).expand(N,M,2),  # [N,2] -> [N,1,2] -> [N,M,2]
box2[:,:2].unsqueeze(0).expand(N,M,2),  # [M,2] -> [1,M,2] -> [N,M,2]
)

rb = torch.min(
box1[:,2:].unsqueeze(1).expand(N,M,2),  # [N,2] -> [N,1,2] -> [N,M,2]
box2[:,2:].unsqueeze(0).expand(N,M,2),  # [M,2] -> [1,M,2] -> [N,M,2]
)

wh = rb - lt  # [N,M,2]
wh[wh<0] = 0  # clip at 0
inter = wh[:,:,0] * wh[:,:,1]  # [N,M]

area1 = (box1[:,2]-box1[:,0]) * (box1[:,3]-box1[:,1])  # [N,]
area2 = (box2[:,2]-box2[:,0]) * (box2[:,3]-box2[:,1])  # [M,]
area1 = area1.unsqueeze(1).expand_as(inter)  # [N,] -> [N,1] -> [N,M]
area2 = area2.unsqueeze(0).expand_as(inter)  # [M,] -> [1,M] -> [N,M]

iou = inter / (area1 + area2 - inter)
return iou

def encode(self, boxes, classes, threshold=0.5):
'''Transform target bounding boxes and class labels 
to SSD boxes and classes.

Match each object box to all the default boxes, pick 
the ones with the
Jaccard-Index > 0.5:
Jaccard(A,B) = AB / (A+B-AB)

Args:
boxes: (tensor) object bounding boxes
 (xmin,ymin,xmax,ymax) of a image, sized [#obj, 4].
classes: (tensor) object class labels of a image, sized [#obj,].
threshold: (float) Jaccard index threshold

Returns:
boxes: (tensor) bounding boxes, sized [#obj, 8732, 4].
classes: (tensor) class labels, sized [8732,]
'''
default_boxes = self.default_boxes
num_default_boxes = default_boxes.size(0)
num_objs = boxes.size(0)

iou = self.iou(  # [#obj,8732]
boxes,
torch.cat([default_boxes[:,:2] - default_boxes[:,2:]/2,
default_boxes[:,:2] + default_boxes[:,2:]/2], 1)
)

iou, max_idx = iou.max(0)  # [1,8732]
max_idx.squeeze_(0)        # [8732,]
iou.squeeze_(0)            # [8732,]

boxes = boxes[max_idx]     # [8732,4]
variances = [0.1, 0.2]
cxcy = (boxes[:,:2] + boxes[:,2:])/2 - default_boxes[:,:2]  # [8732,2]
cxcy /= variances[0] * default_boxes[:,2:]
wh = (boxes[:,2:] - boxes[:,:2]) / default_boxes[:,2:]      # [8732,2]
wh = torch.log(wh) / variances[1]
loc = torch.cat([cxcy, wh], 1)  # [8732,4]

conf = 1 + classes[max_idx]   # [8732,], background class = 0
conf[iou<threshold] = 0       # background
return loc, conf

def nms(self, bboxes, scores, threshold=0.5, mode='union'):
'''Non maximum suppression.

Args:
bboxes: (tensor) bounding boxes, sized [N,4].
scores: (tensor) bbox scores, sized [N,].
threshold: (float) overlap threshold.
mode: (str) 'union' or 'min'.

Returns:
keep: (tensor) selected indices.

Ref:
https://github.com/rbgirshick/py-faster-rcnn/
blob/master/lib/nms/py_cpu_nms.py
'''
x1 = bboxes[:,0]
y1 = bboxes[:,1]
x2 = bboxes[:,2]
y2 = bboxes[:,3]

areas = (x2-x1) * (y2-y1)
_, order = scores.sort(0, descending=True)

keep = []
while order.numel() > 0:
i = order[0]
keep.append(i)

if order.numel() == 1:
break

xx1 = x1[order[1:]].clamp(min=x1[i])
yy1 = y1[order[1:]].clamp(min=y1[i])
xx2 = x2[order[1:]].clamp(max=x2[i])
yy2 = y2[order[1:]].clamp(max=y2[i])

w = (xx2-xx1).clamp(min=0)
h = (yy2-yy1).clamp(min=0)
inter = w*h

if mode == 'union':
ovr = inter / (areas[i] + areas[order[1:]] - inter)
elif mode == 'min':
ovr = inter / areas[order[1:]].clamp(max=areas[i])
else:
raise TypeError('Unknown nms mode: %s.' % mode)

ids = (ovr<=threshold).nonzero().squeeze()
if ids.numel() == 0:
break
order = order[ids+1]
return torch.LongTensor(keep)

def decode(self, loc, conf):
'''Transform predicted loc/conf back to real bbox
 locations and class labels.

Args:
loc: (tensor) predicted loc, sized [8732,4].
conf: (tensor) predicted conf, sized [8732,21].

Returns:
boxes: (tensor) bbox locations, sized [#obj, 4].
labels: (tensor) class labels, sized [#obj,1].
'''
variances = [0.1, 0.2]
wh = torch.exp(loc[:,2:]*variances[1]) * self.default_boxes[:,2:]
cxcy = loc[:,:2] * variances[0] *
 self.default_boxes[:,2:] + self.default_boxes[:,:2]
boxes = torch.cat([cxcy-wh/2, cxcy+wh/2], 1)  # [8732,4]

max_conf, labels = conf.max(1)  # [8732,1]
ids = labels.squeeze(1).nonzero().squeeze(1)  # [#boxes,]

keep = self.nms(boxes[ids], max_conf[ids].squeeze(1))
return boxes[ids][keep], labels[ids][keep], max_conf[ids][keep]
\end{lstlisting}
