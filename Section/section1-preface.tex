\section{前言}
% 正文
在深度学习正式介入之前,传统的目标检测方法都是区域选择提取特征分类回归三部曲,这样就有两个难以解决的问题;其一是区域选择的策略效果差、时间复杂度高;其二是手工提取的特征鲁棒性较差。云计算时代来临后,目标检测算法大家族主要划分为两大派系,一个是R-CNN系两刀流,另一个则是以YOLO为代表的一刀流派下面分别解释一下两刀流和一刀流。 
\begin{uscequation}
	a^2 + b^2 = c^2 
\end{uscequation}
等等。

在深度学习正式介入之前,传统的目标检测方法都是区域选择提取特征分类回归三部曲,这样就有两个难以解决的问题;其一是区域选择的策略效果差、时间复杂度高;其二是手工提取的特征鲁棒性较差。云计算时代来临后,目标检测算法大家族主要划分为两大派系,一个是R-CNN系两刀流,另一个则是以YOLO为代表的一刀流派下面分别解释一下两刀流和一刀流。$a + b = c$,哈哈。


% 图片
\setcounter{figure}{0}
\begin{figure}[h]
	\centering
	\includegraphics{./Pictures/usc.png}	
	\caption{南华大学校徽}	
\end{figure}
在深度学习正式介入之前,传统的目标检测方法都是区域选择提取特征分类回归三部曲,这样就有两个难以解决的问题;其一是区域选择的策略效果差、时间复杂度高;其二是手工提取的特征鲁棒性较差。云计算时代来临后,目标检测算法大家族主要划分为两大派系,一个是R-CNN系两刀流,另一个则是以YOLO为代表的一刀流派下面分别解释一下两刀流和一刀流
% 表格

% 公式
