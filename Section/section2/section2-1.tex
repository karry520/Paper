目标检测是人工智能领域的基础问题\cite{dl},纵观其发展历程,2010年是一个转折点。自2013年\textit{Rich Feature Hierarchies for Accurate Object Detection and Semantic Segmentation}\cite{rcnn}一文的发表,基于卷积神经网络的特征提取代替了传统手工提取特征方法,从此为目标检测领域引入了深度学习这个高效的方法。跟着历史的潮流,本章将简要梳理“目标检测”算法的两种重要思想和扼要地介绍在这些重要思想影响下的那些具有代表性的算法。

概括地说,“目标检测”系列算法宏观上可以划分为两类,一个是以R-CNN\cite{rcnn}为代表的“Two-Stage”方法,另一个则是以YOLO\cite{yolo}为代表的"One-Stage"方法。下面分别解释一下什么是“One-Stage”方法和“Two-Stage”方法。

\textbf{One Stage:}顾名思义,对预测的目标物体直接进行检测,即直接利用位置回归进行目标检测,其特点是简单快速,但是精度要低于“Two Stage”方法,其代表算法是YOLO\cite{yolo}和SSD\cite{ssd}。

\textbf{Two-Stage:}顾名思义,分两步进行目标检测:1、生成可能区域(Region Proposal) \& 利用CNN提取特征。2、放入分类器进行分类,与此同时修正位置。这一流派的算法都离不开Region Proposal,优缺点参半,一方面提高了精度,另一方面却损失了速度,其代表算法是Faster R-CNN\cite{fasterrcnn}。

总的说来,“One-Stage”和“Two-Stage”方法,都是在同一个衡量标准下寻找速度与精度的平衡点。“One-Stage”的方法倾向快,"Two-Stage"的方法更倾向于准。接下来将分别介绍在两种重要思路影响下的具有代表性的算法。

