Viola-Jones人脸检测器之后,在2009年出现了另外一个比较重要的方法:deformable
part model(DPM)\cite{dpm},即可变形部件模型。就人脸检测而言,人脸可以大致看成是一种刚体,通常情况下不会有非常大的形变,比方说嘴巴变到鼻子的位置上去。但是对于其它物体,例如人体,人可以把胳膊抬起来,可以把腿翘上去,这会使得人体有非常多非常大的非刚性变换,而DPM通过对部件进行建模就能够更好地处理这种变换。刚开始的时候大家也试图去尝试用类似于Haar特征+AdaBoost分类器这样的做法来检测行人,但是发现效果不是很好,到2009年之后,有了DPM去建模不同的部件,比如说人有头有胳膊有膝盖,然后同时基于局部的部件和整体去做分类,这样效果就好了很多。DPM相对比较复杂,检测速度比较慢,但是其在人脸检测还有行人和车的检测等任务上还是取得了一定的效果。后来出现了一些加速DPM的方法,试图提高其检测速度。DPM引入了对部件的建模,本身是一个很好的方法,但是其被深度学习的光芒给盖过去了,深度学习在检测精度上带来了非常大的提升,所以研究DPM的一些人也快速转到深度学习上去了。