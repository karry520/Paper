R-CNN 其实是一个很大的家族,自从 rbg 大神发表那篇论文,子孙无数、桃李满天下。在此,我们只探讨 R-CNN 直系亲属,他们的发展顺序如下:

\vspace{10pt}
%%%%%%%%%%%%%%%%%%%%%%%%%%%将要提到glossary文档中%%%%%%%%%%%%%%%%%%%%%%%%%%%%%

\begin{tikzpicture}[node distance=3cm]
\node (rcnn) [Node]
{R-CNN};
\node (sppnet) [Node,right of=rcnn]
{SPP-Net};
\node (fastrcnn) [Node,right of=sppnet]
{Fast R-CNN};
\node (fasterrcnn) [Node,right of=fastrcnn]
{Faster R-CNN};
\node (maskrcnn) [Node,right of=fasterrcnn]
{Mask R-CNN};
%连接具体形状
\draw [arrow] (rcnn) -- (sppnet);
\draw [arrow] (sppnet) -- (fastrcnn);
\draw [arrow] (fastrcnn) -- (fasterrcnn);
\draw [arrow] (fasterrcnn) -- (maskrcnn);
\end{tikzpicture}

他们在整个家族进化的过程中,一致暗埋了一条主线:充分榨干 feature maps 的价值。

\subsubsection{R-CNN}
这个模型,是利用卷积神经网络来做「目标检测」的开山之作,其意义深远不言而喻。
\begin{uscfigure}
	\includegraphics[width=\textwidth]{./Pictures/rcnn-regions_with_cnn_features.png}	
	\caption{RCNN}
\end{uscfigure}

\textbf{解决问题一、速度}
传统的区域选择使用滑窗,每滑一个窗口检测一次,相邻窗口信息重叠高,检测速度慢。R-CNN 使用一个启发式方法(Selective search),先生成候选区域再检测,降低信息冗余程度,从而提高检测速度。

\textbf{解决问题二、特征提取}
传统的手工提取特征鲁棒性差,限于如颜色、纹理等 低层次(Low level)的特征。使用 CNN (卷积神经网络)提取特征,可以提取更高层面的抽象特征,从而提高特征的鲁棒性。

该方法将 PASCAL VOC 上的检测率从 35.1\% 提升到 53.7 \% ,提高了好几个量级。虽然比传统方法好很多,但是从现在的眼光看,只能是初窥门径。

\subsubsection{SPP Net}
R-CNN 提出后的一年,以何恺明、任少卿为首的团队提出了 SPP Net ,这才是真正摸到了卷积神经网络的脉络。也不奇怪,毕竟这些人鼓捣出了 ResNet 残差网络,对神经网络的理解是其他人没法比的。尽管 R-CNN 效果不错,但是他还有两个硬伤:

\textbf{硬伤一、算力冗余}
先生成候选区域,再对区域进行卷积,这里有两个问题:其一是候选区域会有一定程度的重叠,对相同区域进行重复卷积;其二是每个区域进行新的卷积需要新的存储空间。何恺明等人意识到这个可以优化,于是把先生成候选区域再卷积,变成了先卷积后生成区域。“简单地”改变顺序,不仅减少存储量而且加快了训练速度。

\textbf{硬伤二、图片缩放}
无论是剪裁(Crop)还是缩放(Warp),在很大程度上会丢失图片原有的信息导致训练效果不好,如上图所示。直观的理解,把车剪裁成一个门,人看到这个门也不好判断整体是一辆车;把一座高塔缩放成一个胖胖的塔,人看到也没很大把握直接下结论。人都做不到,机器的难度就可想而知了。
\begin{uscfigure}
	\includegraphics[width=\textwidth]{./Pictures/sppnet_crop_warp.png}	
	\caption{RCNN}
\end{uscfigure}
何恺明等人发现了这个问题,于是思索有什么办法能不对图片进行变形,将图片原汁原味地输入进去学习。最后,他们发现问题的根源是 FC Layer(全连接层)需要确定输入维度,于是他们在输入全连接层前定义一个特殊的池化层,将输入的任意尺度 feature maps 组合成特定维度的输出,这个组合可以是不同大小的拼凑,如同拼凑七巧板般。举个例子,我们要输入的维度 64∗256 ,那么我们可以这样组合 32∗256+16∗256+8∗256+8∗256 。
\begin{uscfigure}
	\includegraphics[width=\textwidth]{./Pictures/sppnet_pool_layer.png}	
	\caption{RCNN}
\end{uscfigure}
SPP Net 的出现是如同一道惊雷,不仅减少了计算冗余,更重要的是打破了固定尺寸输入这一束缚,让后来者享受到这一缕阳光。

\subsubsection{Fast R-CNN}
在这篇论文中,引用了 SPP Net 的工作,并且致谢其第一作者何恺明的慷慨解答。纵观全文,最大的建树就是将原来的串行结构改成并行结构
\begin{uscfigure}
	\includegraphics[width=\textwidth]{./Pictures/fast_rcnn.png}	
	\caption{RCNN}	
\end{uscfigure}
原来的 R-CNN 是先对候选框区域进行分类,判断有没有物体,如果有则对 Bounding Box 进行精修 回归。这是一个串联式的任务,那么势必没有并联的快,所以 rbg 就将原有结构改成并行——在分类的同时,对 Bbox 进行回归。这一改变将 Bbox 和 Clf 的 loss 结合起来变成一个 Loss 一起训练,并吸纳了 SPP Net 的优点,最终不仅加快了预测的速度,而且提高了精度。
\subsubsection{Faster R-CNN}
在 Faster R-CNN 前,我们生产候选区域都是用的一系列启发式算法,基于 Low Level 特征生成区域。这样就有两个问题:

第一个问题 是生成区域的靠谱程度随缘,而 两刀流 算法正是依靠生成区域的靠谱程度——生成大量无效区域则会造成算力的浪费、少生成区域则会漏检;

第二个问题 是生成候选区域的算法是在 CPU 上运行的,而我们的训练在 GPU 上面,跨结构交互必定会有损效率。

那么怎么解决这两个问题呢?于是乎,任少卿等人提出了一个 Region Proposal Networks 的概念,利用神经网络自己学习去生成候选区域。
\begin{uscfigure}
	\includegraphics[width=\textwidth]{./Pictures/faster_rcnn.png}	
	\caption{RCNN}	
\end{uscfigure}
这种生成方法同时解决了上述的两个问题,神经网络可以学到更加高层、语义、抽象的特征,生成的候选区域的可靠程度大大提高;可以从上图看出 RPNs 和 RoI Pooling 共用前面的卷积神经网络——将 RPNs 嵌入原有网络,原有网络和 RPNs 一起预测,大大地减少了参数量和预测时间。
\begin{uscfigure}
	\includegraphics[width=\textwidth]{./Pictures/faster_rcnn_anchor.png}	
	\caption{RCNN}	
\end{uscfigure}
在 RPNs 中引入了 anchor 的概念,feature map 中每个滑窗位置都会生成 k 个 anchors,然后判断 anchor 覆盖的图像是前景还是背景,同时回归 Bbox 的精细位置,预测的 Bbox 更加精确。
\subsubsection{Mask R-CNN}
时隔一年,何恺明团队再次更新了 R-CNN 家族,改进 Faster R-CNN 并使用新的 backbone 和 FPN 创造出了 Mask R-CNN 。

\textbf{加一条通道}

我们纵观发展历史,发现 SPP Net 升级为 Fast R-CNN 时结合了两个 loss ,也就是说网络输入了两种信息去训练,结果精度大大提高了。何恺明他们就思考着再加一个信息输入,即图像的 Mask ,信息变多之后会不会有提升呢?于是乎 Mask R-CNN 就这样出来了,不仅可以做「目标检测」还可以同时做「语义分割」,将两个计算机视觉基本任务融入一个框架。没有使用什么 trick ,性能却有了较为明显的提升,这个升级的版本让人们不无啧啧惊叹。作者称其为 meta algorithm ,即一个基础的算法,只要需要「目标检测」或者「语义分割」都可以使用这个作为 Backbone 。


