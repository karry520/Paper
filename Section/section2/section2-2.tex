物体检测在整个计算机领域里,比较成功的一个例子,就是在大概2000年前后出现的Viola-Jones人脸检测器,其使得物体检测相比而言成了一项较为成熟的技术。这个方法基本的思路就是滑动窗口式的,用一个固定大小的窗口在输入图像进行滑动,窗口框定的区域会被送入到分类器,去判断是人脸窗口还是非人脸窗口。滑动的窗口其大小是固定的,但是人脸的大小则多种多样,为了检测不同大小的人脸,还需要把输入图像缩放到不同大小,使得不同大小的人脸能够在某个尺度上和窗口大小相匹配。这种滑动窗口式的做法有一个很明显的问题,就是有太多的位置要去检查,去判断是人脸还是非人脸。

判断是不是人脸,这是两个分类问题,在2000年的时候,采用的是AdaBoost分类器。进行分类时,分类器的输入用的是Haar特征,这是一种非常简单的特征,在图上可以看到有很多黑色和白色的小块,Haar特征就是把黑色区域所有像素值之和减去白色区域所有像素值之和,以这个差值作为一个特征,黑色块和白色块有不同的大小和相对位置关系,这就形成了很多个不同的Haar特征。AdaBoost分类器是一种由多个弱分类器组合而成的强分类器,Viola-Jones检测器是由多个AdaBoost分类器级联组成,这种级联结构的一个重要作用就是加速。

2000年人脸检测技术开始成熟起来之后,就出现了相关的实际应用,例如数码相机中的人脸对焦的功能,照相的时候,相机会自动检测人脸,然后根据人脸的位置把焦距调整得更好。