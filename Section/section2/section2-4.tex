回顾过去,从 YOLO 到 SSD ,人们兼收并蓄把不同思想融合起来。YOLO 使用了分治思想,将输入图片分为 SxS 的网格,不同网格用性能优良的分类器去分类。SSD 将 YOLO 和 Anchor 思想融合起来,并创新使用 Feature Pyramid 结构。但是 Resize 输入,必定会损失许多的信息和一定的精度,这也许是一刀流快的原因。无论如何,YOLO 和 SSD 这两篇论文都是让人不得不赞叹他们想法的精巧,让人受益良多在目标检测中有两个指标:快(Fast) 和 准(Accurate)。一刀流代表的是快,但是最后在快和准中找到了平衡,第一是快,第二是准。两刀流代表的是准,虽然没有那么快但是也有 6 FPS 可接受的程度,第一是准,第二是快。两类算法都有其适用的范围,比如说实时快速动作捕捉,一刀流更胜一筹;复杂、多物体重叠,两刀流当仁不让。没有不好的算法,只有不合适的使用场景。我相信 Mask R-CNN 并不是「目标检测」的最终答案,江山代有才人出,展望未来。