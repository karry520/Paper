\subsubsection{研究背景}
\textbf{自然灾害频发:}根据今日美国报道,美国国家海洋和大气管理局(NOAA)周一宣布,由于三次强大的冰雹,使2017年成为美国遭受冰雹灾害最为严重的年份之一。冰雹灾害让美国遭受了超过3000亿美元的损失。\footnote{美国中文网,2018-1-8}

\textbf{无人机的商用:}从钮扣大小的微型飞行器到可用于执行特殊任务的商用无人机,目前进入市面上的无人机种类和数量都在快速增加。较小的最低10美元就可以买到。消费类无人机的用途主要是娱乐和拍摄,大的可以执行任务的无人机则开始用于商业投递。

\textbf{机器视觉的发展:}近几年深度学习发展迅猛,更是由于前段时间的谷歌的AlphaGo而轰动一时,国内也开始迎来这一技术的研究热潮,深度学习目前还处于发展阶段,不管是理论方面还是实践方面都还有许多问题待解决,不过由于我们处在了一个“大数据”时代,以及计算资源的大大提升,新模型、新理论的验证周期会大大缩短。人工智能时代的开启必然会很大程度的改变这个世界,无论是从交通,医疗,购物,军事等方面,或许我们正处于最好的年代。

\subsubsection{研究意义}
针对传统的房屋瓦片检测方法存在着操作复杂、耗时、高危和具有破坏性等缺点,本项研究尝试利用计算机视觉技术对瓦片进行快速无损检测。本文提出了一种利用基于SSD(Single Shot MultiBox Detector)\cite{ssd}改进算法进行瓦片损害检测的方案,为房屋检测行业提供利用机器视觉处理传统检测问题的高效手段。其意义有三:

\textbf{1、无损检测:}传统的房屋瓦片损害检测工作,需要工作人员爬上房屋拍照,将照片带回工作室进行损害鉴定,这种操作无疑会对瓦片带来人为的破坏;与此同时,工作人员因操作不当受伤甚至致死的报道也时有发生 ,本项研究利用无人机替代工作人员的拍照工作,利用无人机的图像处理模块,实时进行损害的检测,将结果和图片一并送到系统中进行信息的汇总和检测报告的生成。做到的对瓦片和对工作人员的两个“无损”。

\textbf{2、缩短检测周期:}在美国,遇到自然灾害致使房屋受损后,参保的家庭会联系保险公司进行理赔,按照现在的处理水平,一栋房屋平均会耗时7个星期,对于偏远的地方会更久。采用无人机对受损房屋瓦片进行检测会将这个时长缩短到1个星期。大大节约了成本。同时实验也表明有更好的检测效果。

\textbf{3、节约人力成本:}经融危机和通货膨胀造成了人力成本的极大提高。传统的损害检测十分依赖工作人员的经验,所以人力成本一直居高不下,采用搭载了检测算法的无人机进行检测,摆脱了对工作人员经验的依赖,将成本从100美元降到了10美元。