\subsubsection{研究背景}
\textbf{自然灾害频发:}美国中文网根据今日美国报道,美国国家海洋和大气管理局(NOAA)周一宣布,由于三次强大的飓风和凶猛的野火,2017年是美国遭受自然灾害最为严重的一年。这些自然灾害让美国遭受了3060亿美元的损失。

2017年,美国历经16次天气和气候灾害,每一项灾害的损失都超过了10亿美元。总损失约为3060亿美元,创下了新纪录。它打破了2005年的纪录,当年飓风卡特里娜等灾害给美国造成了2150亿美元的损失。NOAA说,去年的灾难共造成全美(包括波多黎各)362人死亡。然而,NOAA气候学家亚当·史密斯(Adam Smith)表示,死亡人数可能会随着波多黎各的后续报告而增加。史密斯说,这也是有史以来破坏最强的飓风季,损失达到2650亿美元。也是是有史以来损失最大的野火季,损失达到了180亿美元。飓风哈维总共造成了1250亿美元的损失,在近30年来造成的破坏仅次于飓风卡特里娜。飓风玛丽亚和艾玛分别造成了900亿美元和500亿美元的损失。这个消息是在德州奥斯汀的美国气象学会年会上公布的。美国大陆和阿拉斯加在2017年的气温也是连续第三年高于平均水平。亚利桑那州、乔治亚州、新墨西哥州、北卡罗莱纳州和南卡罗莱纳州5个州在2017年都经历了有记录以来最温暖的一年。包括阿拉斯加在内的32个州2017年的气温也创下有记录以来的前十高温。\footnote{美国中文网,2018-1-8}

\textbf{无人机的商用:}从手掌大小的微型飞行器到可用于检查输电线路的商用无人机,目前市面上在售的无人机种类和数量都在迅速增加。较小的最低40美元就可以买到,但高端无人机的价格至少也要数千美元(军用无人机的成本更加高昂)。消费类无人机的用途主要是娱乐和拍摄,大的可以执行任务的无人机则开始用于商业投递。

说起无人机技术,大家可以想到层面会是多种多样,有军用无人机技术领域的“全球鹰”、“捕食者”,有民用无人机领域的测绘、航拍甚至快递。那个距离我们最近的“智能科技”,未来发展之路是如何呢?可以说,无人机技术是智能科技皇冠上的一颗璀璨宝石,因为它有最小的体积,集成了最高的人类科技——1981年,第一台商用GPS接收机诞生,重达50磅,价格高达10万美元。现在GPS仅重0.3克,芯片成本也大幅下降,无人机将GPS技术集纳;1976年,柯达推出了第一款数字相机,像素只有10万,重量为3.75磅,价格超过1万美元。而无人机将最新的数字相机技术整合,并根据用途,细分了数种功能性镜头;除此之外,计算机技术、蓝牙通讯技术更是无人机的必备功能,有了这些智能抗美科技的“加持”,说无人机是一枚宝石并不为过。

\textbf{机器视觉的发展:}随着中国制造业的蓬勃发展,机器视觉行业也在中国市场度过了发展的最初时期,不仅国际知名品牌纷纷在中国开展业务,中国本土的企业也逐渐兴起,机器视觉已为广大客户所熟知,应用范围也逐步扩大,由起初的电子制造业和半导体生产企业,发展到了包装,汽车,交通和印刷等多个行业。

近几年深度学习发展迅猛,更是由于前段时间的谷歌的AlphaGo而轰动一时,国内也开始迎来这一技术的研究热潮,深度学习目前还处于发展阶段,不管是理论方面还是实践方面都还有许多问题待解决,不过由于我们处在了一个“大数据”时代,以及计算资源的大大提升,新模型、新理论的验证周期会大大缩短。人工智能时代的开启必然会很大程度的改变这个世界,无论是从交通,医疗,购物,军事等方面,或许我们正处于最好的年代。

\subsubsection{研究意义}
针对传统的房屋瓦片检测方法存在着操作复杂、耗时、高危和具有破坏性等缺点,本项研究尝试利用计算机视觉技术对瓦片进行快速无损检测。本文提出了一种利用基于SSD(Single Shot MultiBox Detector)\cite{ssd}改进算法进行瓦片损害检测的方案,为房屋检测行业提供利用机器视觉处理传统检测问题的高效手段。其意义有三:

\textbf{1、无损检测:}传统的房屋瓦片损害检测工作,需要工作人员爬上房屋拍照,将照片带回工作室进行损害鉴定,这种操作无疑会对瓦片带来人为的破坏;与此同时,工作人员因操作不当受伤甚至致死的报道也时有发生 ,本项研究利用无人机替代工作人员的拍照工作,利用无人机的图像处理模块,实时进行损害的检测,将结果和图片一并送到系统中进行信息的汇总和检测报告的生成。做到的对瓦片和对工作人员的两个“无损”。

\textbf{2、缩短检测周期:}在美国,遇到自然灾害致使房屋受损后,参保的家庭会联系保险公司进行理赔,按照现在的处理水平,一栋房屋平均会耗时7个星期,对于偏远的地方会更久。采用无人机对受损房屋瓦片进行检测会将这个时长缩短到1个星期。大大节约了成本。同时实验也表明有更好的检测效果。

\textbf{3、节约人力成本:}经融危机和通货膨胀造成了人力成本的极大提高。传统的损害检测十分依赖工作人员的经验,所以人力成本一直居高不下,采用搭载了检测算法的无人机进行检测,摆脱了对工作人员经验的依赖,将成本从100美元降到了10美元。