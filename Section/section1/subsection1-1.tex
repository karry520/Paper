\subsubsection{研究背景}
\textbf{自然灾害频发:}根据今日美国报道,美国国家海洋和大气管理局(NOAA)周一宣布,由于三次强大的冰雹,使2017年成为美国遭受冰雹灾害最为严重的年份之一。冰雹灾害让美国遭受了超过3000亿美元的损失。\footnote{美国中文网,2018-1-8}

\textbf{无人机的商用:}从钮扣大小的微型飞行器到可用于执行特殊任务的商用无人机,目前进入市面上的无人机种类和数量都在快速增加。较小的最低10美元就可以买到。消费类无人机的用途主要是拍摄和娱乐,那些可以执行特定任务的无人机则开始用于商业用途。

\textbf{机器视觉的发展:}近年来深度学习发展迅猛,由于谷歌的AlphaGo而轰动一时,深度学习目前还处于发展阶段,理论方面、实践方面都还有许多问题待解决,不过由于我们处在了一个“大数据”时代,以及丰富的计算资源,大大缩短了新模型、新理论的验证周期。人工智能时代的开启必然会很大程度的从人们生活的方方面面改变这个世界。

\subsubsection{研究意义}
针对传统的房屋瓦片检测方法存在着操作复杂、耗时、高危和具有破坏性等缺点,本项研究尝试利用计算机视觉技术对瓦片进行快速无损检测。本文提出了一种利用基于SSD(Single Shot MultiBox Detector)\cite{ssd}改进算法进行瓦片损害检测的方案,为房屋检测行业提供利用机器视觉处理传统检测问题的高效手段。其意义有三:

\textbf{1、无损检测:}传统的房屋瓦片损害检测工作,需要工作人员爬上房屋拍照,将照片带回工作室进行损害鉴定,这种操作无疑会对瓦片带来人为的破坏;与此同时,工作人员因操作不当受伤甚至致死的报道也时有发生 ,本项研究利用无人机替代工作人员的拍照工作,利用无人机的图像处理模块,实时进行损害的检测,将结果和图片一并送到系统中进行信息的汇总和检测报告的生成。做到的对瓦片和对工作人员的两个“无损”。

\textbf{2、缩短检测周期:}在美国,遇到自然灾害致使房屋受损后,参保的家庭会联系保险公司进行理赔,按照现在的处理水平,一栋房屋平均会耗时7个星期,对于偏远的地方会更久。采用无人机对受损房屋瓦片进行检测会将这个时长缩短到1个星期。大大节约了成本。同时实验也表明有更好的检测效果。

\textbf{3、节约人力成本:}经融危机和通货膨胀造成了人力成本的极大提高。传统的损害检测十分依赖工作人员的经验,所以人力成本一直居高不下,采用搭载了检测算法的无人机进行检测,摆脱了对工作人员经验的依赖,将成本从100美元降到了10美元。