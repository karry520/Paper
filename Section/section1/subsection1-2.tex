本文主要是利用SSD改进算法对房屋瓦片损害进行检测,涉及到目标检测系列算法,对于此系列算法的研究是深度学习方向的研究热点。简单的说,目标检测算法是对物体进行定位+分类。目标检测算法与定位算法、分类算法相比,重要的区别是对图片中的对象既要定位又要判断其类别。由于检测到的对象的数量是不定的,所以对象检测的输出长度也是可变的。

\subsubsection{研究现状}
在深度学习被广泛应用于图像特征提取之前,传统的“目标检测”方法通常是区域选择、提取特征、分类回归三个步骤\cite{hkq},这会导致两个难以解决且至关重要的问题;一个是区域选择的策略(Region Proposal)效果不好、时间复杂度特别高;二个是手工设计的特征其鲁棒性较差。随着大数据时代的到来,“目标检测”算法主要形成两种思想。

基于区域预测加分类的目标检测方法,RCNN系列(R-CNN\cite{rcnn}、SPPnet\cite{sppnet}、Fast R-CNN\cite{fastrcnn}以及Faster R-CNN\cite{fasterrcnn})在目标检测精度方面有较好的效果,因为这一类方法将边框预先回归后才传入网络进行训练,所以检测的精度高。这类方法用了两步进行目标检测所以被人们称为“Two Stage”的方法。除此之外,从简化边框回归的角度而产生的以YOLO\cite{yolo}为代表的系列算法,只做了一次边框回归和打分,被人们称为“One Stage”的方法。检测速度快是这类方法的最大特点,虽然YOLO系列算法能达到实时的效果,但是对尺度小的目标训练非常不充分,检测效果不是很理想。换句话说,YOLO系列算法对目标的尺度非常敏感,而且缺少对尺度变化大的物体的泛化能力。

消化吸收了YOLO和Faster R-CNN的优缺点,WeiLiu等人提出了Single Shot MultiBox Detector\cite{ssd}算法,简称为SSD。SSD的改进总体来说有三点:其一、SSD整体设计采取了“One Stage”的思想,以此提高检测速度。其二、网络中融入了Faster R-CNN\cite{fasterrcnn}中的anchors思想,以此提高检测的精度。其三、对feature map分层提取并依次计算边框回归值和分类的操作,以此可以适应多尺度目标的训练和检测任务,解决了“One Stage”系列算法对小目标检测精度不好的问题。SSD的出现是该时期的集大成者,向大家证明了实时高精度目标检测的任务是可以实现的。

\subsubsection{发展难点}
\textbf{对象的数量是不确定的(Variable Number of Objects)}。在设计网络模型时,一般要将数据约定为大小固定的向量。由于图片中对象的数量事先是未知的,所以我们没办法预先约定好输出的维度。这无疑会增加模型的复杂性。

\textbf{对象检测窗口的调整(Resizing)}。另一个棘手的问题是不同对象其大小也是不同的,意思是不论是大的对象还是小到只有几个像素大小的对象,都希望能有较好的检测效果。利用多尺度的滑动窗口可以解决多尺度对象的问题,但是效率很低。

\textbf{建模}。最后的挑战是一个模型要同时解决两个截然不同的需求——定位和分类。
