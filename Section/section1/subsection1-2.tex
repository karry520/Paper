本文主要是利用SSD改进算法对房屋瓦片损害进行检测,涉及到目标检测系列算法,对于此系列算法的研究是深度学习方向的研究热点。简单的说,目标检测算法是对物体进行定位+分类。目标检测算法与定位算法、分类算法相比,重要的区别是对图片中的对象既要定位又要判断其类别。且对象检测的输出长度是可变的,因为检测到的对象的数量是不定的。

\subsubsection{研究现状}
在深度学习正式介入之前,传统的“目标检测”方法都是区域选择、提取特征、分类回归三部曲\cite{hkq},这样就有两个难以解决且至关重要的问题;其一是区域选择的策略效果差、时间复杂度高;其二是手工提取的特征鲁棒性较差。大数据时代来临后,“目标检测”算法大家族主要划分为两大派系。

基于”Proposal + Classification”的Object Detection的方法,RCNN系列(R-CNN\cite{rcnn}、SPPnet\cite{sppnet}、Fast R-CNN\cite{fastrcnn}以及Faster R-CNN\cite{fasterrcnn})取得了非常好的效果,因为这一类方法先预先回归一次边框,然后再进行骨干网络训练,所以精度要高,这类方法被称为“Two Stage”的方法。但也正是由于此,这类方法在速度方面还有待改进。由此,YOLO\cite{yolo}应运而生,YOLO系列只做了一次边框回归和打分,所以相比于RCNN系列被称为“One Stage”的方法,这类方法的最大特点就是速度快。但是YOLO虽然能达到实时的效果,但是由于只做了一次边框回归并打分,这类方法导致了小目标训练非常不充分,对于小目标的检测效果非常的差。简而言之,YOLO系列对于目标的尺度比较敏感,而且对于尺度变化较大的物体泛化能力比较差。

针对YOLO和Faster R-CNN的各自不足与优势,WeiLiu等人提出了Single Shot MultiBox Detector,简称为SSD。SSD整个网络采取了“One Stage”的思想,以此提高检测速度。并且网络中融入了Faster R-CNN\cite{fasterrcnn}中的anchors思想,并且做了特征分层提取并依次计算边框回归和分类操作,由此可以适应多种尺度目标的训练和检测任务。SSD的出现是该时期的集大成者,向大家证明了实时高精度目标检测的可行性。

\subsubsection{发展难点}
\textbf{对象的数量可变(Variable Number of Objects)}。在训练模型时,通常需要将数据表示为固定大小的向量。但是,由于图片中对象的数量事先不知道,所以我们不知道正确的输出维度。因此需要一些后期处理,这增加了模型的复杂性。

\textbf{对象检测窗口的调整(Resizing)}。另一个巨大的挑战是各种可能的对象大小,即在进行分类时,既希望占图片大部分的对象进行分类,又想要找到一些可能只有几个像素、或者是原始图像一小部分的小对象。使用不同尺寸的滑动窗口可以解决这个问题,但效率很低。

\textbf{建模}。第三个挑战是同时解决两个问题——定位和分类如何用一个简单的模型解决这两种截然不同的需求。
