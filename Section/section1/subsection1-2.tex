\subsubsection{研究现状}
定位和分类可以迭代起来,最终在一张图片汇总对多个对象进行检测和分类。对象检测是在图像上发现和分类一个变量的问题。对象检测与定位、分类相比,重要的区别是这个“变量”。对象检测的输出长度是可变的,因为检测到的对象的数量会根据图像的不同而变化。在本文中,我们将深入了解对象检测的实际应用、作为机器学习的对象检测的主要问题是什么、以及深度学习如何在这几年里解决这个问题。

\subsubsection{发展难点}
\textbf{可变数量的对象 (Variable number of objects)}我们之前提到了关于对象数量可变的问题,但我们却没讲它为什么是一个问题。在训练机器学习模型时,通常需要将数据表示为固定大小的向量。但是,由于图片中对象的数量事先不知道,所以我们不知道正确的输出维度。因此需要一些后期处理,这增加了模型的复杂性。一般使用滑动窗口的方法来处理可变数量的对象,通过滑动固定大小的窗口,在所有的地方生成固定大小的特征。在得到这些被过滤后的特征之后,一些被丢弃,另一些被合并以生成最终的结果。
这里有个滑动窗口的例子:滑动窗口的动图。

\textbf{调整对象检测窗口大小 (Resizing)}另一个巨大的挑战是各种可能的对象大小,即在进行分类时,既希望占图片大部分的对象进行分类,又想要找到一些可能只有12个像素、或者是原始图像一小部分的小对象。使用不同尺寸的滑动窗口可以解决这个问题,但效率很低。

\textbf{建模}第三个挑战是同时解决两个问题——如何用一个简单的模型解决两种不同的需求,即定位和分类。
