“特征分层提取,并依次对其边框进行回归和分类”是SSD主体网络的设计思想。SSD算法理论上对不同尺度的目标都有很好的适应性,不同层次的feature map表示了不同layer的语义信息,low layer的feature map能代表low leayer的语义信息(含有能提高语义分割质量的细节信息),low layer的图像适合小尺度目标的学习。high layer的feature map能代表high layer语义信息,起到光滑边缘的作用,适合对大尺度的目标进行学习。

对不同尺度的目标进行学习的全过程共分成6个阶段,每个阶段都能得到一个特征图,然后对该特征图里的目标进行边框回归和分类。

第一阶段:VGG16\cite{vgg16}网络的前5层卷积网络;

第二、三阶段:VGG16中的fc6和fc7两个全连接层转化为两个卷积层Conv6和Conv7;

第四至六阶段:Conv8、Conv9、Conv10和Conv11四层网络;

如图\ref{ssd}所示就是SSD的网络结构。在每个学习阶段,网络包含了多个以小卷积组成的卷积层。


