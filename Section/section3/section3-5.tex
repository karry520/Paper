\subsubsection{损失函数-Focus Loss}
我们知道object detection的算法主要可以分为两大类:two-stage detector和one-stage detector。前者是指类似Faster RCNN,RFCN这样需要region proposal的检测算法,这类算法可以达到很高的准确率,但是速度较慢。虽然可以通过减少proposal的数量或降低输入图像的分辨率等方式达到提速,但是速度并没有质的提升。后者是指类似YOLO,SSD这样不需要region proposal,直接回归的检测算法,这类算法速度很快,但是准确率不如前者。作者提出focal loss的出发点也是希望one-stage detector可以达到two-stage detector的准确率,同时不影响原有的速度。

既然有了出发点,那么就要找one-stage detector的准确率不如two-stage detector的原因,作者认为原因是:样本的类别不均衡导致的。我们知道在object detection领域,一张图像可能生成成千上万的candidate locations,但是其中只有很少一部分是包含object的,这就带来了类别不均衡。那么类别不均衡会带来什么后果呢?引用原文讲的两个后果:(1) training is inefficient as most locations are easy negatives that contribute no useful learning signal; (2) en masse, the easy negatives can overwhelm training and lead to degenerate models. 什么意思呢?负样本数量太大,占总的loss的大部分,而且多是容易分类的,因此使得模型的优化方向并不是我们所希望的那样。其实先前也有一些算法来处理类别不均衡的问题,比如OHEM(online hard example mining),OHEM的主要思想可以用原文的一句话概括:In OHEM each example is scored by its loss, non-maximum suppression (nms) is then applied, and a minibatch is constructed with the highest-loss examples。OHEM算法虽然增加了错分类样本的权重,但是OHEM算法忽略了容易分类的样本。

因此针对类别不均衡问题,作者提出一种新的损失函数:focal loss,这个损失函数是在标准交叉熵损失基础上修改得到的。这个函数可以通过减少易分类样本的权重,使得模型在训练时更专注于难分类的样本。为了证明focal loss的有效性,作者设计了一个dense detector:RetinaNet,并且在训练时采用focal loss训练。实验证明RetinaNet不仅可以达到one-stage detector的速度,也能有two-stage detector的准确率。
\subsubsection{非极大抑制-Soft-NMS}
NMS算法的大致过程可以看原文这段话:First, it sorts all detection boxes on the basis of their scores. The detection box M with the maximum score is selected and all other detection boxes with a significant overlap (using a pre-defined threshold) with M are suppressed. This process is recursively applied on the remaining boxes.那么传统的NMS算法存在什么问题呢?可以看Figure1。在Fiugre1中,检测算法本来应该输出两个框,但是传统的NMS算法可能会把score较低的绿框过滤掉(如果绿框和红框的IOU大于设定的阈值就会被过滤掉),导致只检测出一个object(一个马),显然这样object的recall就比较低了。 
可以看出NMS算法是略显粗暴(hard),因为NMS直接将和得分最大的box的IOU大于某个阈值的box的得分置零,那么有没有soft一点的方法呢?这就是本文提出Soft NMS。那么Soft-NMS算法到底是什么样呢?简单讲就是:An algorithm which decays the detection scores of all other objects as a continuous function of their overlap with M. 换句话说就是用稍低一点的分数来代替原有的分数,而不是直接置零。另外由于Soft NMS可以很方便地引入到object detection算法中,不需要重新训练原有的模型,因此这是该算法的一大优点。
\subsubsection{数据增强}
采用数据扩增(Data Augmentation)可以提升SSD的性能,主要采用的技术有水平翻转(horizontal flip),随机裁剪加颜色扭曲(random crop \& color distortion),随机采集块域(Randomly sample a patch)(获取小目标训练样本)