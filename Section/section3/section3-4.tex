相比于基于“Region Proposal“方法,SSD需要精确的将Ground Truth与输出结果相对应。这样才能做到检测准确率的提高。

\textbf{匹配策略}

\line

Default boxes生成器

Hard Negative Mining

Data Augmentation

\line

\textcircled{1}\textbf{匹配策略}

此处的匹配指的是Ground Truth与Default Box的匹配。SSD采取的方法与Faster R-CNN中的做法相似。操作分成两步:第一步是根据Ground Truth与Default Box最大的交叠(Overlap)将二者进行匹配,找到IOU最大的区域作为正样本;第二步是与Overlap大于某个阈值的Ground truth进行匹配。

\textcircled{2}\textbf{Default Boxes生成器}

\textcircled{3}\textbf{Hard Negative Mining}
3.4.2节探讨了“One Stage”方法检测精度低的原因,经过上述匹配策略只有少量的正样本,会得到大量的负样本,从而导致了正负样本数量不平衡问题。经过试验表明,正负样本的不均衡会导致检测正确率低下\cite{focal-loss}。因此在训练过程中采用了Hard Negative Mining\cite{hnm}的策略,根据所有的box的Confidence Loss的排序结果,将正负样本的比例控制在1:3以下,实验结果表明,这样做后能提高4\%左右的准确度。

\textcircled{4}\textbf{Data Augmentation}

对不同尺度目标的输入数据进行了增强处理,增加了模型的鲁棒性。

\textcircled{1}使用整张图像作为输入

\textcircled{2}使用IOU和目标物体为0.1、0.3、0.5、0.7和0.9的patch,这些patch在原图的大小的[0.1, 1]之间,相应的宽高比在[1/2, 2]之间。

\textcircled{3}随机采取一个Patch

