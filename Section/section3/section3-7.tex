\subsubsection{SSD中采用的损失函数}
\textbf{联合LOSS FUNCTION:}SSD网络对于每个Stage输出的特征图都进行边框回归和分类操作。SSD网络中作者设计了一个联合损失函数
\[
	L(x,c,l,g) = \frac{1}{N}(L_{conf}(x,c) + \alpha L_{loc}(x,l,g)
\]
其中:

 1、$L_{conf}$代表的是分类误差,采用的是多分类的softmax loss
 
 2、$L_{loc}$代表的是回归误差,采用的是Smooth L1 loss
 
 3、$\alpha$取1
 
 回归loss,smoothL1:
\begin{align*}
 L_loc(x,l,g) &= \sum _{i \in Pos} ^{N} \sum_{m \in \{cx,cy,w,h\}} x_{ij}^k smooth _{L1} (l_i ^m - g_j ^m) \\
 g_j^{cx} &= \left( g_j ^{cx} - d_i ^{cx} \right) / d_i ^w \\
 
\end{align*}

\subsubsection{改进损失函数-Focus Loss}
我们知道object detection的算法主要可以分为两大类:two-stage detector和one-stage detector。前者是指类似Faster RCNN,RFCN这样需要region proposal的检测算法,这类算法可以达到很高的准确率,但是速度较慢。虽然可以通过减少proposal的数量或降低输入图像的分辨率等方式达到提速,但是速度并没有质的提升。后者是指类似YOLO,SSD这样不需要region proposal,直接回归的检测算法,这类算法速度很快,但是准确率不如前者。作者提出focal loss的出发点也是希望one-stage detector可以达到two-stage detector的准确率,同时不影响原有的速度。

既然有了出发点,那么就要找one-stage detector的准确率不如two-stage detector的原因,作者认为原因是:样本的类别不均衡导致的。我们知道在object detection领域,一张图像可能生成成千上万的candidate locations,但是其中只有很少一部分是包含object的,这就带来了类别不均衡。那么类别不均衡会带来什么后果呢?引用原文讲的两个后果:

(1) training is inefficient as most locations are easy negatives that contribute no useful learning signal; 

(2) en masse, the easy negatives can overwhelm training and lead to degenerate models. 

什么意思呢?负样本数量太大,占总的loss的大部分,而且多是容易分类的,因此使得模型的优化方向并不是我们所希望的那样。其实先前也有一些算法来处理类别不均衡的问题,比如OHEM(online hard example mining),OHEM的主要思想可以用原文的一句话概括:In OHEM each example is scored by its loss, non-maximum suppression (nms) is then applied, and a minibatch is constructed with the highest-loss examples。OHEM算法虽然增加了错分类样本的权重,但是OHEM算法忽略了容易分类的样本。因此针对类别不均衡问题,作者提出一种新的损失函数:focal loss,这个损失函数是在标准交叉熵损失基础上修改得到的。这个函数可以通过减少易分类样本的权重,使得模型在训练时更专注于难分类的样本。为了证明focal loss的有效性,作者设计了一个dense detector:RetinaNet,并且在训练时采用focal loss训练。实验证明RetinaNet不仅可以达到one-stage detector的速度,也能有two-stage detector的准确率。

\textbf{标准交叉熵损失函数}

原来的分类loss是各个训练样本交叉熵的直接求和,也就是各个样本的权重是一样的。如下图所示:

\begin{equation}
CE(p,y)=\left\{
\begin{aligned}
& -\log (p) 	&if y = 1\\
& -\log (1-p) 	&otherwise
\end{aligned}
\right.
\end{equation}

其中:

CE表示cross entropy,p表示预测样本属于1的概率,y表示label,y的取值为{+1,-1},这里仅仅以二分类为例,多分类以此类推。为了表示简便,我们用$p_t$表示样本属于true class的概率。所以上式可以写成

\[
	CE(p,y) = CE(p_t) = -\log (p_t)
\]

\textbf{平衡交叉熵}

既然"one stage detector"在训练的时候正负样本的数量差距很大,那么一种常见的做法就是给正负样本加上权重,负样本出现的频次多,那么就降低负样本的权重,正样本数量少,就相对提高正样本的权重,如下式所示:
\[
	CE(p_t) = -\alpha \log (p_t)
\]

\textbf{focal-loss定义}

作者实际上解决的是easy example和hard example不均衡的问题,这个和训练时候正负样本不均衡是两码事,因为正负样本里面都会有简单的样本和容易分错的样本。作者提出的focal loss,相当于是对各个样本加上了各自的权重,这个权重是和网络预测该样本属于true class的概率相关的,显然,如果网络预测的该样本属于true class的概率很大,那么这个样本对网络来说就属于easy(well-classified) example。如果网络预测的该样本属于true class的概率很小,那么这个样本对网络来说就属于hard example。为了训练一个有效的classification part,显然应该降低绝大部分easy example的权重,相对增大hard example的权重。作者提出的focal loss如下式所示: 
\[
	FL(p_t) = -(1 - p_t)^\gamma \log (p_t)
\]
参数$\gamma$大于0。当参数$\gamma = 0$的时候,就是普通的交叉熵,作者的实验中发现$\gamma = 2$效果最高。可以看到,当γ一定的时候,比如等于2,一样easy example($p_t = 0.9$)的loss要比标准的交叉熵loss小100+倍,当$p_t=0.968$时,要小1000+倍,但是对于hard example($p_t < 0.5$),loss最多小了4倍。这样的话hard example的权重相对就提升了很多。实际实验中,作者和3.2一样,对正负样本又做了一个reweighting 
\[
	FL(p_t) = -\alpha _t(1 - p_t)^\gamma \log (p_t)
\]
